\documentclass[../seminar.tex]{subfiles}


\begin{document}

\subsection{Qualitative Comparison}
	
\begin{flushleft}
Two competing technologies provide virtual resources to run NFV workloads – Virtual Machines and Containers. [1] (TABLE 1) provides a qualitative comparison of these technologies. Below table [Table 1] is an expansion of the same.

\begin{tabular}{ ||p{4cm}|p{4cm}|p{4cm}|| }
	
\hline
\textbf{Feature} & \textbf{Virtualization} & \textbf{Containerization} \\
\hline\hline
Compute Virtualization & Hardware Abstraction & Application Binary Interface \\
\hline
Interaction with the System & Hypervisor-based & System Calls \\
\hline
Base Workload & Complete Guest OS & Processes and Dependencies \\
\hline
Provisioning & Slow and Complex & Fast and Scalable \\
\hline
Resource Consumption & High & Low \\
\hline
Isolation & Full Isolation & Process-Level Isolation \\
\hline
Management & Openstack, AWS, Azure, GCE & Kubernetes, Docker Swarm, Mesos \\
\hline
Single Independent Abstraction & VM Instances & Multi-container PODs \\
\hline\hline
\end{tabular}
	
\end{flushleft}

\subsection{Quantitative Comparison}

\begin{flushleft}
	


\end{flushleft}
\end{document}